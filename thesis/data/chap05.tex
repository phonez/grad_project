% !TeX root = ../thuthesis-example.tex

\chapter{结论与展望}


综上,本文的主要工作有如下几点:
其一, 针对Rosetta软件,实现了非天然氨基酸params文件和rotlib文件的自动化生成,为将来应用Rosetta实现更大规模的结构筛选打下基础。
其二,选取九种含有代表性侧链结构的非天然氨基酸,以及PDB数据库中包含该非天然氨基酸的蛋白质,通过位点突变和局部的结构优化,完成Rosetta对这些非天然氨基酸的建模测试。
结果表明:以ref2015作为打分函数,在当前优化策略下,除α-取代的其它非天然氨基酸,Rosetta能够基本正确建模;对于侧链含有芳香基团的情形,Rosetta建模较好。
其三,根据以上流程,以SETD8复合物为例,通过多肽不同位点的突变计算,挑选出四种有利于蛋白-多肽结合的突变方式,从定性的角度获得了与实验相符的结果。

本文对非天然氨基酸的建模测试依赖于包含该非天然氨基酸的蛋白质结构,但是PDB数据库中包含非天然氨基酸的蛋白质晶体结构数量较少,这限制了测试所能覆盖的侧链结构种类。
建模测试中,只抹除了非天然氨基酸的坐标信息,而保留了周围环境的坐标信息,不能模拟真实的设计情形(即周围环境的坐标信息同样是未知的)。
为实现更加严格的测试,可以考虑将该位点突变成其它氨基酸,优化结构(从而一定程度上抹除了周围环境的坐标信息);再突变回待研究的非天然氨基酸,优化结构。
并且,Rosetta数据库中给出的氨基酸结构与蛋白质中的氨基酸结构在原子坐标上会存在一定偏差,这一偏差可能导致Rosetta无法正确建模侧链结构中带有环烷烃的非天然氨基酸。
针对此问题,一个可能的解决方案是通过RDKit等工具包搜索非天然氨基酸更多可能的构象,并将其视为不同的结构导入至Rosetta数据库中。
另外,本文在对非天然氨基酸的建模过程中,没有对Rosetta的能量项进行修改,比如针对不同种类的非天然氨基酸给出解折叠能量项的计算方法。
近期,Mulligan等人\cite{RN109}试图在环肽抑制剂的设计过程中引入非天然氨基酸,但最终未能实现很好的建模。这说明Rosetta在处理非天然氨基酸方面仍存在短板。

在后续非天然多肽的再设计中,可以考虑引入更加精细的结构优化方法。
比如,使用Cartesian\_ddG方法\cite{RN114}预测点突变对结合能力的影响,并拓展该方法的应用范围,使之兼容非天然氨基酸。
又如,引入RotamerTrialsMinMover\cite{RN110},在能量最小化前完成对每个非天然氨基酸的侧链搜寻;
以及通过多肽骨架的整体移动跳出局部的构象搜索,增大采样空间;等等。
更进一步,尝试在没有复合物结构的情况下,通过FlexPepDock等\cite{RN115}方法实现多肽的从头设计。
Mulligan等人的工作可以作为一个值得学习的参考范例。
