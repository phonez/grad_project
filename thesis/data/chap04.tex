% !TeX root = ../thuthesis-example.tex

\chapter{非天然多肽的再设计}



\section{SETD8复合物}

进一步,根据已有的多肽-蛋白质复合物结构,计算模拟在多肽不同位点挑选非天然氨基酸的过程,以得到结合能力最强的多肽。
为此,我们通过文献调研,选取甲基转移酶SETD8及其与H4肽段构成的复合物(PDB:1ZKK)作为研究对象,如图~\ref{fig:complex}。
Judge等人通过在肽段中不同位点引入不同种类的非天然氨基酸,实验筛选得到含有正亮氨酸的非共价结合的多肽抑制剂(PDB:5TEG)。
本文中,我们希望通过Rosetta计算非天然氨基酸突变对蛋白-多肽结合能力的影响,完成对非天然多肽的再设计,
并对比Judge等人得到的实验结果,评估计算设计方案。
\begin{figure}
  \centering
  \includegraphics[width=0.5\linewidth]{1ZKK.png}
  \caption{甲基转移酶SETD8复合物}
  \label{fig:complex}
\end{figure}


\section{Rosetta打分值与蛋白-多肽结合能力的关系的理论推导}

Rosetta中的打分值反映体系某一状态下的能量,因此,我们可以通过突变前后打分的差值反映突变前后蛋白-多肽复合物的稳定性变化,进而计算多肽位点的突变对其结合能力的影响。
具体而言,将多肽与蛋白质的结合过程表示为图~\ref{fig:bind}。
\begin{figure}
  \centering
  \includegraphics[width=0.7\linewidth]{bind.jpg}
  \caption{多肽与蛋白质的结合过程}
  \label{fig:bind}
\end{figure}

此例中,由于位点突变发生在多肽上,目标蛋白氨基酸序列不变,因此结合过程的吉布斯自由能变可以表达为
\begin{equation}
  \increment G(bind)=G_{comp}-G_{pro}-G_{pep}
  \label{eq:G_before_mutate}
\end{equation}
\begin{equation}
  \increment G^{'}(bind)=G_{comp}^{'}-G_{pro}-G_{pep}^{'}
  \label{eq:G_after_mutate}
\end{equation}

于是位点突变对结合能力的影响可以用突变后反应吉布斯自由能变与突变前反应吉布斯自由能变的差来表示,以$ddG$表示这一差值,即
\begin{equation}
  ddG=\increment G^{'}(bind)-\increment G(bind)
\end{equation}

依据反应吉布斯自由能变和平衡常数(此处即为解离常数$K_d$的倒数)的关系
\begin{align}
  \increment G(bind) &= -RTln(\frac{1}{K_d}) \\
                     &= RTln(K_d)
\end{align}

综合以上几式,可以得到
\begin{equation}
  ddG=RTln(\frac{K_{d}^{'}}{K_d})
\end{equation}

另一方面,以Rosetta中的打分值替换式~\eqref{eq:G_before_mutate}和式~\eqref{eq:G_after_mutate}中的吉布斯自由能,有
\begin{align}
  \increment\increment t &= (t_{comp}^{'}-t_{pro}^{'}-t_{pep}^{'})-(t_{comp}-t_{pro}-t_{pep}) \\
                         &= (t_{comp}^{'}-t_{pep}^{'})-(t_{comp}-t_{pep}) \\
                         &= \increment t^{'}-\increment t
\end{align}

于是,我们将打分值与多肽与蛋白质结合的解离常数关联起来,从理论上推导得到二者的数学关系式
\begin{equation}
  \increment\increment t\propto ln(\frac{K_{d}^{'}}{K_d})
  \label{eq:relationship}
\end{equation}


\section{设计方案}
对于复合物,参照前文给出的优化策略,第一步,在内坐标下对复合物做限制性优化,以输出的50个结构中打分最低者作为下一步的初始结构;
第二步,对非天然氨基酸周围\SI{12}{Å}的局部环境做放松处理。
由于蛋白-多肽复合物的结构已知,在第二步中我们考虑较简单的情形,只打开侧链自由度,即突变后只采样侧链而不采样骨架。
这一步输出200个结构,取打分最低的三个复合物结构的打分平均值。
需要指出的是,在对复合物的建模过程中,我们没有处理复合物中的腺苷甲硫氨酸小分子(图~\ref{fig:complex}中以球棍表示的分子),
比如为该分子准备相应的params文件、定义结合口袋等等。
这是由于在优化过程和后续分析中,我们假定分子骨架基本不变,并通过对复合物结构和天然结构二者的打分值作差,
以期消除小分子本身带来的打分上的误差。
换言之,只评价小分子与多肽互作对复合物的打分贡献,而不考虑小分子的打分绝对值。

对于多肽,以复合物中的多肽结构作为初始结构,对整体做放松处理。输出100个结构,取打分最低的三条多肽的打分平均值。
参考Judge等人的工作,额外选取如下几种非天然氨基酸(如图~\ref{fig:mutant}),并分别在多肽指定位点进行突变。
需要指出的是,对于在多肽第17位或第22位突变的情形,其初始结构为经过两步优化后、多肽第20位已经突变为正亮氨酸的打分最低者;
对于突变成环己烷取代的丙氨酸衍生物ALC,考虑了平伏键取代和直立键取代两种构象,取打分较低者作为最终结果。
\begin{figure}
  \centering
  \includegraphics[width=0.51\linewidth]{mutant.pdf}
  \caption{用以突变的部分非天然氨基酸}
  \label{fig:mutant}
\end{figure}



\section{实验结果}

不同位点突变结果如表~\ref{tab:mutate}。
\begin{table}
  \centering
  \begin{threeparttable}[c]
    \caption{多肽的突变结果}
    \label{tab:mutate}
    \begin{tabular}{cccccc}
      \toprule
      突变位点                & $t_{comp}$ (REU)    &  $t_{pep}$ (REU) & $\increment t$ (REU) & $\increment\increment t$ (REU) & $K_d$ ($\mu M$)\tnote{①} \\
      \midrule      
      WT                      & -469.67               & -0.275             & -469.390          & 0                       & 46.6                     \\     
      \rowcolor{red!20!green!40!}
      K20MET                  & -469.61               & 3.152              & -472.757          & -3.367                  & 1.26                     \\  
      \rowcolor{red!20!green!40!}
      K20NLE                  & -471.87               & -0.603             & -471.263          & -1.873                  & 0.14                     \\  
      \rowcolor{red!20!green!40!}
      L22HLX, K20NLE          & -472.21               & -1.435             & -470.778          & -1.388                  & 0.21                     \\          
      \rowcolor{red!20!green!40!}
      L22OG2, K20NLE          & -473.61               & -3.367             & -470.246          & -0.856                  & 0.21                     \\          
      R17MET, K20NLE          & -469.08               & -0.129             & -468.948          & 0.442                   & 14.8                     \\         
      R17OG2, K20NLE          & -470.34               & -2.496             & -467.839          & 1.551                   & 3.04                     \\         
      L22ILE, K20NLE          & -465.96               & -0.452             & -465.508          & 3.882                   & n/a                      \\        
      L22ALC, K20NLE\tnote{②} & -463.22               & 1.748              & -464.970          & 4.42                    & 0.12                     \\       
      L220A1, K20NLE          & -472.65               & -7.874             & -464.778          & 4.612                   & 0.32                     \\         
      R17ILE, K20NLE          & -465.27               & -1.259             & -464.015          & 5.375                   & 5.57                     \\         
      R17HLX, K20NLE          & -467.56               & -4.083             & -463.475          & 5.915                   & 1.84                     \\         
      L22DZU, K20NLE          & -468.36               & -5.909             & -462.454          & 6.936                   & 0.8                      \\        
      L22OAS, K20NLE          & -463.29               & -0.908             & -462.378          & 7.012                   & 4.64                     \\         
      K206QX                  & -456.81               & -1.208             & -455.600          & 13.79                   & 1.58                     \\          
      K20TRP                  & -448.34               & 1.849              & -450.193          & 19.197                  & n/a                      \\         
      K20TIH                  & -448.58               & -3.243             & -445.339          & 24.051                  & 6.7                      \\          
      K20PPN                  & -439.78               & -1.267             & -438.509          & 30.881                  & n/a                      \\          
      K20200                  & -435.96               & -1.422             & -434.542          & 34.848                  & n/a                      \\          
      \bottomrule
    \end{tabular}
    \begin{tablenotes}
      \item [①] 解离常数的实验数据源自文献,文献中部分突变位点的数值未给出,以n/a代替。
      \item [②] 对于22位突变成ALC的情形,直立键取代构象在复合物中打分更低,平伏键构象在柔性多肽中打分更低。
    \end{tablenotes}
  \end{threeparttable}
\end{table}
依照公式~\eqref{eq:relationship}所给出的关系,以天然多肽的打分差值、解离常数作为基准,进行线性拟合(不考虑未给出解离常数的突变结果),
如图~\ref{fig:linear_fit}。
\begin{figure}
  \centering
  \includegraphics[width=0.49\linewidth]{linear_fit.pdf}
  \caption{突变结果的线性拟合}
  \label{fig:linear_fit}
\end{figure}

如图所示,线性拟合的结果并不理想。误差主要来自于:对于复合物或多肽的结构优化方法太过粗糙,不够准确。
此处,我们调用Rosetta中的FastRelax方法[33, 34],直接对局部结构做放松处理。
由于FastRelax方法只能寻找到初始结构附近的能量最小点,对于柔性多肽而言,采样不够彻底,难以寻找到全局的能量最小点。
对于复合物而言,位点突变时不考虑骨架二面角的变动以及骨架整体平移能否对蛋白-多肽的结合造成更加有利的能量贡献,
且只对突变位点周围\SI{12}{Å}的残基做侧链重排,无疑缩小了采样空间,为打分值引入了误差。
并且,在内坐标下的侧链采样并不改变氨基酸内部原子的相对位置,即不改变键长、键角等值,这与真实情况存在一定偏差。
另一方面,在所有突变中,假定多肽的初始骨架均相同也是较为粗糙的。
天然状态下,从多肽单体到形成复合物的过程中,伴随着多肽构象的连续调整,
而多肽某一位点氨基酸侧链的大小、疏水性、带电量等因素的改变,
可能在能量曲线中形成较大的势垒,最终导致无法形成甚至远离我们假定的初始结构。

虽然整体的实验结果在拟合公式~\eqref{eq:relationship}上表现较差,但是如果我们仅考虑$\increment\increment t$的正负性,仍能得到较为满意的结果。
$\increment\increment t$为正,表示突变使得蛋白-多肽的结合更加不稳定;$\increment\increment t$为负,表示突变使得蛋白-多肽的结合更加稳定。
在表~\ref{tab:mutate}中,K20MET,K20NLE,L22HLX、K20NLE,L22OG2、K20NLE四种突变方式均为蛋白-多肽的结合带来有利的影响。
定性地看,这与文献中实验给出的结论一致。