% !TeX root = ../thuthesis-example.tex

\chapter{引言}



\section{非天然氨基酸}

非天然氨基酸,通常用以指代非典型氨基酸(noncanonical amino acid,NCAA),意指那些不在经典的二十种氨基酸范围内的其它氨基酸。
从描述的对象上来看,非典型氨基酸也包含一部分天然的组成蛋白质的氨基酸,
最常见的如侧链中包含各种天然修饰的氨基酸,包括甲基化修饰的赖氨酸、精氨酸,磷酸化修饰的丝氨酸、苏氨酸等等;
又如,在某些生物体内存在的硒代半胱氨酸、吡咯赖氨酸\cite{RN116}。
尽管这与“非天然”的语义存在冲突,但在本文中我们不区分二者的差别。

关于非天然氨基酸,主要包括两个方向的研究:扩展可遗传编码的非天然氨基酸库;探索非天然氨基酸在生命科学中的可能应用。
非天然氨基酸的遗传编码技术最早由Schultz等人\cite{RN81}开发。
他们通过突变筛选出生物正交的氨酰-tRNA合成酶及相应的tRNA,开创性地将非天然氨基酸引入到蛋白质中\cite{RN82}。
目前,已经能够在原核生物和真核生物中编码超过150种具有不同结构或功能的非天然氨基酸\cite{RN79}。
这一技术为科学家们理解生命体系和改造生物分子带来了更多可能。
在理解生命体系方面,例如,
Coin等人\cite{RN83}通过在促肾上腺皮质激素释放因子受体蛋白中插入对叠氮苯丙氨酸和对氟乙酰基苯丙氨酸,鉴定出该蛋白与神经肽配体的结合位点;
Cellitti等人\cite{RN84}通过引入NMR信号明显的对三氟甲氧基苯丙氨酸,显著降低了蛋白质的NMR解谱难度;
Wang等人\cite{RN85}通过引入光致脱笼的酪氨酸,实现了时间分辨的蛋白质激活。
在改造生物分子方面,例如,
Mills等人\cite{RN41}通过引入联吡啶取代的丙氨酸衍生物,借助Rosetta计算设计出具有金属结合能力的蛋白质;
Drienovská等人\cite{RN86}通过引入对氨基苯丙氨酸作为亲核的催化残基,设计出催化形成腙和肟的酶。
这一系列事实表明,非天然氨基酸作为化学生物学工具箱中一类有力的工具,
能够帮助加深人们对分子结构机制的认知,
拓宽人们在蛋白质工程领域的研究视野,
启发人们创造新的分子调控手段。



\section{靶向蛋白质的非天然多肽}

如前所述,非天然氨基酸具有广阔的应用前景。这启发我们将其应用到蛋白质靶向的分子开发当中。
发展靶向蛋白质的方法学,其重要性自不必多言。
根据目标蛋白的结构信息,目前已经发展了一系列具有特异性识别能力的结合分子(protein binder)。
按照它们的分子大小可以将其粗略地划分为:蛋白质(抗体),多肽,超分子(如杯芳烃,南瓜环等等)\cite{RN88},化学小分子(小分子药物)。
其中,多肽分子易于合成、改造,并且可能兼具蛋白质与化学小分子的优势,即前者的高选择性、高效性,后者的稳定性、低免疫原性\cite{RN89};
这也使其成为近几年来药物领域的热点研究对象\cite{RN90}。
于是,我们选择从柔性多肽(flexible peptide)出发,
尝试通过多肽某个位点上的非天然氨基酸突变(文中,以“非天然多肽”代表那些包含有非天然氨基酸的多肽链)为之带来更强的结合能力或更高的选择性。

事实上,在设计多肽结合分子(主要为多肽抑制剂)的过程中,非天然氨基酸作为一种选择,早已进入科学家的视野。
比如针对X染色体连锁的凋亡抑制蛋白设计的非共价多肽拮抗剂\cite{RN91,RN92,RN93}或共价多肽抑制剂\cite{RN94},
针对含半胱氨酸的天冬氨酸蛋白水解酶(caspase)家族某一成员设计的选择性共价多肽抑制剂\cite{RN95,RN96},等等。
在这些案例中,
从多肽结构上看,既有限制性多肽(constrained peptide),如环肽、订书肽(stapled peptide),也有柔性多肽;
从多肽与蛋白质的结合方式上看,既有共价结合,也有非共价结合。
但是,就设计流程而言,研究人员往往需要经历大规模的实验筛选,以获取到理想的多肽结合分子;这一过程盲目且费时费力。
即使能够根据蛋白质-多肽复合物的结构信息为某个位点非天然氨基酸的选择提供一些经验性的指导,
这样的筛选依然显得缺乏系统性、程序性,不利于推广至更多目标蛋白的非天然多肽设计当中。
另外,这些非天然多肽的设计出发点大都是基于对已有的多肽结合分子的改造,无疑限制了这种设计思路的进一步发展。
因此,我们期待一种更加理性的、更加程序性的方式,以实现非天然多肽的高效设计。



\section{建模工具Rosetta}

Rosetta是由Baker等人在二十世纪九十年代中期开始研发的一套软件,用以研究蛋白质结构预测和蛋白质折叠相关问题。
经过这些年的不断努力,Rosetta目前已经发展成为适用于多种生物分子体系、多种生物学问题的软件套件。
就生物分子体系而言,Rosetta能兼容抗体、膜蛋白、多肽、核酸等不同类型的生物分子;
就生物学问题而言,Rosetta已被广泛应用于蛋白质结构预测、蛋白质分子对接(如蛋白质-蛋白质对接,蛋白质-多肽对接,蛋白质-配体对接等)、蛋白质设计等问题的研究当中\cite{RN97}。
近些年来,与Rosetta相关的科学成果层出不穷。
比如,在近几年蛋白质结构预测的关键评估测试(Critical Assessment of protein Structure Prediction,CASP)中,Rosetta均取得了不错的成绩\cite{RN98}。
在实际应用方面,
Cui等人\cite{RN99}近期通过Rosetta Enzyme Design重新设计天冬氨酸酶,报道了用于不对称氢胺化的C-N裂解酶,并以此实现了多种非天然氨基酸的高效合成,展现了Rosetta在工业生产中的应用价值。
Chevalier等人\cite{RN100}针对流感病毒血凝素,从设计的两万多个微蛋白中筛选得到高亲和力的结合分子;这些分子表现出低免疫原性和较好的治疗保护作用。
Cao等人\cite{RN101}针对新冠病毒刺突蛋白中与人ACE2受体蛋白的结合域,设计出半抑制浓度为皮摩尔量级的抑制剂。
这两个例子展现了人们在应对医疗问题时通过Rosetta给出的解决方案。
在基础研究方面,
Langan等人\cite{RN3}报道了具有生物活性的蛋白质开关的从头设计,
Chen等人\cite{RN10}报道了蛋白质逻辑门的从头设计,为蛋白质的功能调控带来新的视角。
另外, Xu等人\cite{RN102}、Vorobieva等人\cite{RN103}近期分别报道了跨膜蛋白的从头设计,为理解与膜蛋白相关的生物物理原理提供参考。
从以上几个例子可以看出,Rosetta在面对多种蛋白质设计问题时均展现了不错的建模能力。

为将非天然氨基酸引入Rosetta中,Renfrew等人\cite{RN71}最早建立了一套系统的工作流程,
并在引入正缬氨酸、高丝氨酸等非天然氨基酸的同时,针对钙蛋白酶及其抑制蛋白的结合界面进行重新设计;
通过计算模拟和荧光偏振结合分析实验,他们鉴定得到:将610位点的苯丙氨酸突变为4-甲基苯丙氨酸能够为抑制蛋白与钙蛋白酶的结合能力带来两倍的提升。
在此之后,出现一些将非天然氨基酸引入蛋白质设计中的精彩案例。
除上文所述的金属蛋白质设计外,
Pearson、Mills等人\cite{RN104}通过引入联苯取代的丙氨酸衍生物,从π-π堆积作用、侧链形状互补性等多方面精心设计该非天然氨基酸周围的“微环境”,
成功捕获到非天然氨基酸处于联苯构象过渡态时的蛋白晶体结构。
这些例子鼓舞我们借助Rosetta这一套强大的生物建模软件,实现非天然多肽的理性设计。



\section{本文研究内容和意义}

本文借助Rosetta实现了非天然多肽的理性设计,研究内容主要包括两个部分。
其一,针对Rosetta软件,构建了一套将任意种类的非天然氨基酸添加到Rosetta非天然氨基酸数据库中的自动化流程,以丰富序列设计的多样性,扩大结构搜寻的空间;
同时,从PDB数据库中筛选了部分含有非天然氨基酸的蛋白质,以测试Rosetta对非天然氨基酸的物理建模能力。
在经典范围内仅有的二十种氨基酸其结构和功能均是有限的,难以满足多样的计算设计需要。
而正如前文所述,非天然氨基酸在多种问题背景下的应用潜力被不断发掘;
并且从理论上看,非天然氨基酸的种类是无限的,能够极大地拓展结构设计的边界。
尽管Rosetta数据库中内置了部分非天然氨基酸,自动化地构建针对任意种类的非天然氨基酸数据库、系统性地测试Rosetta对非天然氨基酸的建模能力仍是有意义的。
一方面,它能够为将来更大规模的结构设计奠定基础;
另一方面,能为后续进一步完善Rosetta对非天然氨基酸的建模提出指导性意见。
其二,在蛋白-多肽复合物结构的基础上,对多肽序列进行再设计。


