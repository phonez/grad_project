% !TeX root = ../thuthesis-example.tex

\chapter{非天然氨基酸数据库的构建}

\section{数据库及其结构}

数据库中记录的非天然氨基酸主要来源于商业可购买的非天然氨基酸以及PDB数据库中整理得到的部分非天然氨基酸。
对于每一种非天然氨基酸,提供不含立体化学层的国际化合物标识码(InChI)和相应的SMILES字符串,用以记录不含手性信息的化学结构。
同时整理了非天然氨基酸的商业信息,包括价格、CAS编号、InChI立体化学层等等。
我们使用MySQL管理该数据库,目前已统计得到136种侧链结构互不相同的非天然氨基酸。



\section{构建流程}

为将非天然氨基酸添加至Rosetta数据库中,需要为该非天然氨基酸准备params文件和rotlib文件。
params文件记录了内坐标空间下各个原子的坐标、原子之间的成键情况、侧链二面角(chi)的个数等等;
rotlib文件则记录了非天然氨基酸在不同骨架二面角下侧链的优势旋转角。其基本流程如下图~\ref{fig:workflow}。
首先,需要在该非天然氨基酸的C端添加N-甲基化修饰、N端添加乙酰化修饰。
通过构造这样一个“二肽”结构,模拟非天然氨基酸在蛋白骨架中的行为,以生成rotlib文件。
我们使用RDKit化学工具包\cite{gRN118}完成非天然氨基酸的C端修饰和N端修饰。
其次,使用Gaussian软件\cite{RN113}对该“二肽”结构进行结构优化。选择泛函B3LYP,基组6-31G,色散校正项GD3BJ。
随后,指定内坐标下的中心原子序号(骨架中N原子的序号,也即Rosetta中原子生成树的根),使用OpenBabel软件\cite{RN117}转换文件格式,得到params文件。
最后,为各个侧链二面角指定可能的优势旋转角,作为Rosetta内置MakeRotLib方法中K-means聚类算法的初始中心点。
\begin{figure}
  \centering
  \includegraphics[width=1.0\linewidth]{workflow.pdf}
  \caption{params文件和rotlib文件的构建流程}
  \label{fig:workflow}
\end{figure}

