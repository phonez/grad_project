% !TeX root = ../thuthesis-example.tex

% 中英文摘要和关键字

\begin{abstract}
  本文针对Rosetta软件,构建了一套将非天然氨基酸整合至数据库中的自动化流程。
  并且,针对几种具有代表性侧链结构的非天然氨基酸,通过对蛋白质的位点突变和结构优化,完成了Rosetta对这些非天然氨基酸的建模测试。
  最终我们得到:除α-取代的非天然氨基酸外,Rosetta对其它大多数非天然氨基酸基本能够建模正确,对侧链中含有芳香基团的建模较好。
  在这一观点的支撑下,本文从甲基转移酶SETD8以及H4天然肽段构成的复合物出发,通过不同位点的突变扫描,
  挑选出四种有利于蛋白-多肽结合的突变,从定性角度获得了与实验相符的结果。

  % 关键词用“英文逗号”分隔,输出时会自动处理为正确的分隔符
  \thusetup{
    keywords = {Rosetta, 非天然氨基酸, 非天然多肽, 蛋白质靶向},
  }
\end{abstract}

\begin{abstract*}
  This article builds an automated process for the integration of noncanonical amino acids into the database for Rosetta software.
  In addition, for several noncanonical amino acids with representative side chain structures, we complete the modeling test of these noncanonical amino acids through site mutation and structural optimization of the proteins in Rosetta.
  In the end, we get: except for α-substituted noncanonical amino acids, Rosetta can basically model most other noncanonical amino acids correctly, and can model noncanonical amino acids with aromatic groups in the side chain well.
  With the support of this point of view, this article starts from the complex composed of methyltransferase SETD8 and H4 natural peptides, and selects four mutations that are conducive to protein-peptide binding through mutation scanning at different sites.
  From a qualitative perspective, a result consistent with the experiment is obtained.

  % Use comma as seperator when inputting
  \thusetup{
    keywords* = {Rosetta, noncanonical amino acid, unnatural peptide, protein-targeting},
  }
\end{abstract*}
